\chapter{Conclusion et perspectives}
\label{chap:con}

Se base sur la méthode SVM, une méthode très populaire dans le domaine d'apprentissage automatique, les auteurs ont développé Pegasos \cite{sss07} qui utilise SGD binaire. En générale, la méthode dans Pegasos n'est pas meilleure que la méthode SVM sur la classification car on peut comprendre SGD est une version simple de SVM où l'on ne doit pas résoudre le problème de programme de quadratique. Se base sur SGD binaire, nous avons développé notre algorithme pour que SGD s'adapte bien au problème de multi-classes. Pour le problème de classification de multi-classes avec one-vs-all, la version dans Pegasos trouve le problème de balance. Pour cette raison, nous avons modifié la méthode dans Pegasos. Dans cette recherche, nous ne faisons le point sur le résultat de classification, mais sur la vitesse de classification sur les bases d'images réelles. Donc, SGD s'adapte bien.\\

Bien que SGD a des avantages de la vitesse, cette méthode soit difficile de tester, particulièrement, le choix des paramètres entrées tel que le nombre d'itération, lamda. Dans l'avenir, nous étudierons pour chercher des paramètres optimales de la méthode. Nous testerons aussi des bases d'images plus grandes tel que ImageNet. Nous développerons pour SGD s'adapte aux autres domaines, tel que à la classification de vidéos.
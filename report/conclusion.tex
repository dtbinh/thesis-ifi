\addcontentsline{toc}{chapter}{Conclusion et perspectives}
\chapter*{Conclusion et perspectives}
\label{chap:con}
Se basant sur la méthode SVM, une méthode très populaire dans le domaine d'apprentissage automatique, les auteurs ont développé Pegasos \cite{sss07} utilisant le SGD binaire. En générale, la méthode Pegasos n'est pas meilleure que la méthode SVM sur la classification car le SGD est une version simple du SVM où l'on ne doit pas résoudre le problème de programme quadratique. Se basant sur SGD binaire, nous avons développé notre algorithme pour que SGD s'adapte bien au problème de multi-classes. Nous avons implémenté avec toutes les deux options possibles one-vs-one et one-vs-all. Pour le problème de classification de multi-classes avec one-vs-all, le SGD Pegasos trouve le problème de balancé. Pour cette raison, nous avons fait la échantillonnage et l'équilibra de la méthode SGD Pegasos. Dans cette recherche, nous ne faisons pas le point sur le résultat de classification, mais sur la vitesse de classification sur les bases d'images réelles. Donc, le MC-SGD s'adapte bien. Pour étudier notre algorithme, nous avons développé MC-SGD-Toy.\\

Bien que le MC-SGD a des avantages de la vitesse, cette méthode soit difficile de choisir des paramètres entrées tel que le nombre d'itération, lamda. Dans l'avenir, nous étudierons pour chercher des paramètres optimales de la méthode. Nous testerons aussi des bases d'images plus grandes tel que ImageNet. Nous développerons pour le SGD s'adapte aux autres domaines, tel que à la classification de vidéos.
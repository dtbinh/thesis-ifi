\clearpage
\thispagestyle{plain}
\chapter*{Résumé}

Dans ce projet, nous avons étudié les problèmes concernant la classification d'images et développé un algorithme parallèle multi-classes basé sur la descente de gradient stochastique. Dans un premier temps, nous avons étudié la représentation des images par descripteurs locaux SIFT (Scalable Invariant Feature Transform) \cite{low99}. D'abord, on extrait des SIFT à partir des images. À la fin de cette étape, nous obtenons un ensemble des vecteurs caractéristiques pour chaque image. L'étape suivante consiste à construire un vocabulaire visuel en appliquant un algorithme de clustering (e.g. k-means) sur un ensemble des vecteurs caractéristiques. Un cluster correspond à un mot visuel. Enfin, une image se représente par un histogramme des mots visuels. Cette approche s'inspire au modèle sac-de-mots largement utilisé dans l'analyse des données textuelles. Dans un second temps, nous nous concentrons sur le problème d'apprentissage automatique basé sur la descente de gradient stochastique \cite{sss07}. Pour améliorer la vitesse de l'algorithme sur des machines de multi-cœurs, nous avons aussi parallélisé cet algorithme en utilisant la libraire OpenMP. Pour balancer la base d'apprentissage, nous avons développé une version balancée de SGD pour adapter à la classification multi-classes.


